% Copyright (c) 2013 Datalogisk Kantineforening.
% Licenseret under EUPL, version 1.1 udelukkende.
% Licensteksten: https://joinup.ec.europa.eu/software/page/eupl/licence-eupl

\documentclass{article}

\usepackage{skiltefabrikken}

\begin{document}

\hovedoverskrift{Vejledning til KEN}

\maketitle

\fontsize{17.28}{18}\selectfont

\begin{enumerate}

\setcounter{enumi}{-1}

\bfseries \item Skyl servicen grundigt af \normalfont

  \begin{itemize}

  \item Husk at fjerne alle mad/kafferester.  KEN er en
    desinficeringsmaskine, ikke en opvaskemaskine.

  \end{itemize}

\bfseries \item Stil det afskyllede service i bakken ved vasken
\normalfont

\begin{itemize}

  \item Kopper, skåle og andre konkave objekter skal vende nedad,
    således at de ikke bliver fyldt med vand.

  \item Bestik skal i en bestikholder.

  \item Lad vær med at stille opvasken i lag -- så bliver det ikke rent!

  \end{itemize}

\bfseries \item Hvis bakken er fyldt: \normalfont
  \begin{enumerate}

  \item Tag den rene bakke ud af KEN og stil den på den rene side ved
    tallerkenerne.

  \item Sæt den beskidte bakke i KEN og start ham ved at trykke på
    knappen ved pilen.

  \item Find en tom bakke fra over kopperne og sæt den frem.

  \end{enumerate}

\bfseries \item Hvis der i forvejen står bakker på den rene side: \normalfont

  \begin{itemize}

  \item Sæt tingene og bakken på plads.

  \end{itemize}

\bfseries \item Hvis du er den sidste der går: \normalfont

  \begin{itemize}

    \item Udfør en tømning af KEN (se vejledning på guiden, som ligger oven på
      KEN)

  \end{itemize}

\end{enumerate}

\underskriv

\end{document}
